\documentclass[promaster]{thesis-uestc}

\title{中文题目}{English Title} % 论文题目
\author{姓名}{English Name} % 作者姓名
\setdate[submit]{2022年3月17日} % 论文提交日期,可留空
\setdate[oral]{2022年4月15日}  % 答辩日期,可留空
\setdate[confer]{2022年6月8日} % 学位授予日期,可留空
\advisor{导师姓名\chinesespace 导师职称}{English name English title}
\coAdvisor{合作导师姓名\chinesespace 导师职称}{Co advisor English name English title} % 仅专业硕士/博士使用,在扉页/英文首页添加合作导师
\school{计算机科学与工程学院(网络空间安全学院)}{School of Computer Science and Engineering(School of Cyberspace Security)} % 学院信息
\major{计算机科学与技术}{Computer Science and Technology} % 专业信息
\studentnumber{xxxxx} % 学号
\ProfessionalDegreeArea{随便学学} % 专业硕士专用:专业学位领域
\ClassificationNumber{TP309.2} % 分类号
\ClassifiedClass{公开} % 密级
\UDCNumber{004.78} % UDC号

% 取消注释以下内容,用于禁止文中换行处的英语单词自动截断换行。
% \tolerance=1
% \emergencystretch=\maxdimen
% \hyphenpenalty=10000
% \hbadness=10000

\makeglossaries % 产生缩略词表/符号表专用,不使用时请注释
\newacronym[description=逻辑卷管理器]{lvm}{LVM}{Logical Volume Manager} % 定义缩略词/符号:以本项为例,逻辑卷管理器为中文名称;lvm用于文内引用;LVM为显示的应为缩略语或符号;Logical Volume Manager为显示的英文全称/描述

\begin{document}
\makecover % 封面+中英文扉页
\originalitydeclaration % 原创新声明
% \signatureofdeclaration{signature.pdf} % 用于添加扫描版签字后的原创新声明(使用时取消注释本行,并注释掉上一行)
% 中文摘要
\begin{chineseabstract}

    \chinesekeyword{xxx,xxx,xxx} % 中文关键词
\end{chineseabstract}
% 英文摘要
\begin{englishabstract}

    \englishkeyword{xxx, xxx, xxx} % 英文关键词
\end{englishabstract}

\thesistableofcontents % 目录
\thesisfigurelist % 图目录,仅在需要时添加,一般情况下请注释
\thesistablelist % 表目录,仅在需要时添加,一般情况下请注释
% \glsaddall % 默认仅显示被正文引用的项,取消注释以显示所有已定义的缩略词/符号
\thesisglossarylist % 缩略词表,仅在需要时添加,一般情况下请注释
\thesissymbollist % 符号表,仅在需要时添加,一般情况下请注释

% 正文内容
\chapter{绪\hspace{6pt}论}
角标参考文献\citing{chen2001hao},\cite{clerc2010discrete}做了xxx。\gls{lvm}
\chapter{相关研究基础}
\chapter{工作内容xxx}
\chapter{全文总结与展望}

\thesisacknowledgement

xxxx % 直接填写致谢内容,写法与正文一致

\thesisbibliography[large]{reference} % 参考文献

\thesisappendix
\chapter{xxxx} % 直接填写附录内容,写法与正文一致

% 攻读学位期间成果(本科不添加),例如:
\begin{thesistheaccomplish}
    \section{学术论文}
    \bibitem{SGXDedup} \textbf{Ren, Yanjing} and Li, Jingwei and Yang, Zuoru and Lee, Patrick PC and Zhang, Xiaosong. Accelerating Encrypted Deduplication via SGX[C]. Proc.of USENIX ATC, 2021, 957-971. \textbf{CCF-A}
    \section{发明专利}
    \bibitem{CN111338572B} 李经纬, 杨祚儒, \textbf{任彦璟}, 李柏晴, 张小松. 一种可调节加密重复数据删除方法:CN111338572B[P]. 2021-09-14.
\end{thesistheaccomplish}

% 以下为本科同学专用,插入文献翻译
\thesistranslationoriginal
\section{The OFDM Model of Multiple Carrier Waves}
% \insertPDFPage{} % 用于插入单页PDF文件,例如原始文献(该操作会导致页码被取消,请谨慎使用)

\thesistranslationchinese
\section{基于多载波索引键控的正交频分多路复用系统模型}

\end{document}